
% Default to the notebook output style




% Inherit from the specified cell style.


\author{Jeremy Margolin, Michael Klyachman, Liam Schumm, \\ Elias O'Malley, Jules Goduco-Bunting}

\newcommand{\fullline}{\noindent\rule{\textwidth}{0.3pt}}

\documentclass[11pt]{article}


    \usepackage[T1]{fontenc}
    % Nicer default font (+ math font) than Computer Modern for most use cases
    \usepackage{mathpazo}

    % Basic figure setup, for now with no caption control since it's done
    % automatically by Pandoc (which extracts ![](path) syntax from Markdown).
    \usepackage{graphicx}
    % We will generate all images so they have a width \maxwidth. This means
    % that they will get their normal width if they fit onto the page, but
    % are scaled down if they would overflow the margins.
    \makeatletter
    \def\maxwidth{\ifdim\Gin@nat@width>\linewidth\linewidth
    \else\Gin@nat@width\fi}
    \makeatother
    \let\Oldincludegraphics\includegraphics
    % Set max figure width to be 80% of text width, for now hardcoded.
    \renewcommand{\includegraphics}[1]{\Oldincludegraphics[width=.8\maxwidth]{#1}}
    % Ensure that by default, figures have no caption (until we provide a
    % proper Figure object with a Caption API and a way to capture that
    % in the conversion process - todo).
    \usepackage{caption}
    \DeclareCaptionLabelFormat{nolabel}{}
    \captionsetup{labelformat=nolabel}

    \usepackage{adjustbox} % Used to constrain images to a maximum size
    \usepackage{xcolor} % Allow colors to be defined
    \usepackage{enumerate} % Needed for markdown enumerations to work
    \usepackage{geometry} % Used to adjust the document margins
    \usepackage{amsmath} % Equations
    \usepackage{amssymb} % Equations
    \usepackage{textcomp} % defines textquotesingle
    % Hack from http://tex.stackexchange.com/a/47451/13684:
    \AtBeginDocument{%
        \def\PYZsq{\textquotesingle}% Upright quotes in Pygmentized code
    }
    \usepackage{upquote} % Upright quotes for verbatim code
    \usepackage{eurosym} % defines \euro
    \usepackage[mathletters]{ucs} % Extended unicode (utf-8) support
    \usepackage[utf8x]{inputenc} % Allow utf-8 characters in the tex document
    \usepackage{fancyvrb} % verbatim replacement that allows latex
    \usepackage{grffile} % extends the file name processing of package graphics
                         % to support a larger range
    % The hyperref package gives us a pdf with properly built
    % internal navigation ('pdf bookmarks' for the table of contents,
    % internal cross-reference links, web links for URLs, etc.)
    \usepackage{hyperref}
    \usepackage{longtable} % longtable support required by pandoc >1.10
    \usepackage{booktabs}  % table support for pandoc > 1.12.2
    \usepackage[inline]{enumitem} % IRkernel/repr support (it uses the enumerate* environment)
    \usepackage[normalem]{ulem} % ulem is needed to support strikethroughs (\sout)
                                % normalem makes italics be italics, not underlines
    \usepackage{mathrsfs}



    % Colors for the hyperref package
    \definecolor{urlcolor}{rgb}{0,.145,.698}
    \definecolor{linkcolor}{rgb}{.71,0.21,0.01}
    \definecolor{citecolor}{rgb}{.12,.54,.11}

    % ANSI colors
    \definecolor{ansi-black}{HTML}{3E424D}
    \definecolor{ansi-black-intense}{HTML}{282C36}
    \definecolor{ansi-red}{HTML}{E75C58}
    \definecolor{ansi-red-intense}{HTML}{B22B31}
    \definecolor{ansi-green}{HTML}{00A250}
    \definecolor{ansi-green-intense}{HTML}{007427}
    \definecolor{ansi-yellow}{HTML}{DDB62B}
    \definecolor{ansi-yellow-intense}{HTML}{B27D12}
    \definecolor{ansi-blue}{HTML}{208FFB}
    \definecolor{ansi-blue-intense}{HTML}{0065CA}
    \definecolor{ansi-magenta}{HTML}{D160C4}
    \definecolor{ansi-magenta-intense}{HTML}{A03196}
    \definecolor{ansi-cyan}{HTML}{60C6C8}
    \definecolor{ansi-cyan-intense}{HTML}{258F8F}
    \definecolor{ansi-white}{HTML}{C5C1B4}
    \definecolor{ansi-white-intense}{HTML}{A1A6B2}
    \definecolor{ansi-default-inverse-fg}{HTML}{FFFFFF}
    \definecolor{ansi-default-inverse-bg}{HTML}{000000}

    % commands and environments needed by pandoc snippets
    % extracted from the output of `pandoc -s`
    \providecommand{\tightlist}{%
      \setlength{\itemsep}{0pt}\setlength{\parskip}{0pt}}
    \DefineVerbatimEnvironment{Highlighting}{Verbatim}{commandchars=\\\{\}}
    % Add ',fontsize=\small' for more characters per line
    \newenvironment{Shaded}{}{}
    \newcommand{\KeywordTok}[1]{\textcolor[rgb]{0.00,0.44,0.13}{\textbf{{#1}}}}
    \newcommand{\DataTypeTok}[1]{\textcolor[rgb]{0.56,0.13,0.00}{{#1}}}
    \newcommand{\DecValTok}[1]{\textcolor[rgb]{0.25,0.63,0.44}{{#1}}}
    \newcommand{\BaseNTok}[1]{\textcolor[rgb]{0.25,0.63,0.44}{{#1}}}
    \newcommand{\FloatTok}[1]{\textcolor[rgb]{0.25,0.63,0.44}{{#1}}}
    \newcommand{\CharTok}[1]{\textcolor[rgb]{0.25,0.44,0.63}{{#1}}}
    \newcommand{\StringTok}[1]{\textcolor[rgb]{0.25,0.44,0.63}{{#1}}}
    \newcommand{\CommentTok}[1]{\textcolor[rgb]{0.38,0.63,0.69}{\textit{{#1}}}}
    \newcommand{\OtherTok}[1]{\textcolor[rgb]{0.00,0.44,0.13}{{#1}}}
    \newcommand{\AlertTok}[1]{\textcolor[rgb]{1.00,0.00,0.00}{\textbf{{#1}}}}
    \newcommand{\FunctionTok}[1]{\textcolor[rgb]{0.02,0.16,0.49}{{#1}}}
    \newcommand{\RegionMarkerTok}[1]{{#1}}
    \newcommand{\ErrorTok}[1]{\textcolor[rgb]{1.00,0.00,0.00}{\textbf{{#1}}}}
    \newcommand{\NormalTok}[1]{{#1}}

    % Additional commands for more recent versions of Pandoc
    \newcommand{\ConstantTok}[1]{\textcolor[rgb]{0.53,0.00,0.00}{{#1}}}
    \newcommand{\SpecialCharTok}[1]{\textcolor[rgb]{0.25,0.44,0.63}{{#1}}}
    \newcommand{\VerbatimStringTok}[1]{\textcolor[rgb]{0.25,0.44,0.63}{{#1}}}
    \newcommand{\SpecialStringTok}[1]{\textcolor[rgb]{0.73,0.40,0.53}{{#1}}}
    \newcommand{\ImportTok}[1]{{#1}}
    \newcommand{\DocumentationTok}[1]{\textcolor[rgb]{0.73,0.13,0.13}{\textit{{#1}}}}
    \newcommand{\AnnotationTok}[1]{\textcolor[rgb]{0.38,0.63,0.69}{\textbf{\textit{{#1}}}}}
    \newcommand{\CommentVarTok}[1]{\textcolor[rgb]{0.38,0.63,0.69}{\textbf{\textit{{#1}}}}}
    \newcommand{\VariableTok}[1]{\textcolor[rgb]{0.10,0.09,0.49}{{#1}}}
    \newcommand{\ControlFlowTok}[1]{\textcolor[rgb]{0.00,0.44,0.13}{\textbf{{#1}}}}
    \newcommand{\OperatorTok}[1]{\textcolor[rgb]{0.40,0.40,0.40}{{#1}}}
    \newcommand{\BuiltInTok}[1]{{#1}}
    \newcommand{\ExtensionTok}[1]{{#1}}
    \newcommand{\PreprocessorTok}[1]{\textcolor[rgb]{0.74,0.48,0.00}{{#1}}}
    \newcommand{\AttributeTok}[1]{\textcolor[rgb]{0.49,0.56,0.16}{{#1}}}
    \newcommand{\InformationTok}[1]{\textcolor[rgb]{0.38,0.63,0.69}{\textbf{\textit{{#1}}}}}
    \newcommand{\WarningTok}[1]{\textcolor[rgb]{0.38,0.63,0.69}{\textbf{\textit{{#1}}}}}


    % Define a nice break command that doesn't care if a line doesn't already
    % exist.
    \def\br{\hspace*{\fill} \\* }
    % Math Jax compatibility definitions
    \def\gt{>}
    \def\lt{<}
    \let\Oldtex\TeX
    \let\Oldlatex\LaTeX
    \renewcommand{\TeX}{\textrm{\Oldtex}}
    \renewcommand{\LaTeX}{\textrm{\Oldlatex}}
    % Document parameters
    % Document title
    \title{Scenario Response Topic 1- Automobile Insurance}
\usepackage{fancyvrb}
\usepackage{xparse}

    \ExplSyntaxOn
    \box_new:N \l_fvrb_box
    \tl_new:N \l_fvrb_tl

    \RenewDocumentCommand \FancyVerbFormatLine { m }
     {
       \hbox_set:Nn \l_fvrb_box { #1 }
        \dim_compare:nNnTF { \box_wd:N \l_fvrb_box }>{ \linewidth }
          {%box to big
           \fvrb_use_linecontents:n { #1 }
          }
          {%box fits
           \box_use:N \l_fvrb_box
          }
     }

    \cs_new:Npn \fvrb_use_linecontents:n  #1
     {
      \group_begin:
       \vbox_set:Nn \l_fvrb_box
         {\hsize=\linewidth\parindent0pt
           \advance\hsize by -2em
           \hspace*{-2em}
            #1
         }
        \hspace*{2em}\box_use:N \l_fvrb_box
      \group_end:
    }

    \ExplSyntaxOff

    \usepackage{xparse}



    % Pygments definitions

\makeatletter
\def\PY@reset{\let\PY@it=\relax \let\PY@bf=\relax%
    \let\PY@ul=\relax \let\PY@tc=\relax%
    \let\PY@bc=\relax \let\PY@ff=\relax}
\def\PY@tok#1{\csname PY@tok@#1\endcsname}
\def\PY@toks#1+{\ifx\relax#1\empty\else%
    \PY@tok{#1}\expandafter\PY@toks\fi}
\def\PY@do#1{\PY@bc{\PY@tc{\PY@ul{%
    \PY@it{\PY@bf{\PY@ff{#1}}}}}}}
\def\PY#1#2{\PY@reset\PY@toks#1+\relax+\PY@do{#2}}

\expandafter\def\csname PY@tok@gd\endcsname{\def\PY@tc##1{\textcolor[rgb]{0.63,0.00,0.00}{##1}}}
\expandafter\def\csname PY@tok@gu\endcsname{\let\PY@bf=\textbf\def\PY@tc##1{\textcolor[rgb]{0.50,0.00,0.50}{##1}}}
\expandafter\def\csname PY@tok@gt\endcsname{\def\PY@tc##1{\textcolor[rgb]{0.00,0.27,0.87}{##1}}}
\expandafter\def\csname PY@tok@gs\endcsname{\let\PY@bf=\textbf}
\expandafter\def\csname PY@tok@gr\endcsname{\def\PY@tc##1{\textcolor[rgb]{1.00,0.00,0.00}{##1}}}
\expandafter\def\csname PY@tok@cm\endcsname{\let\PY@it=\textit\def\PY@tc##1{\textcolor[rgb]{0.25,0.50,0.50}{##1}}}
\expandafter\def\csname PY@tok@vg\endcsname{\def\PY@tc##1{\textcolor[rgb]{0.10,0.09,0.49}{##1}}}
\expandafter\def\csname PY@tok@vi\endcsname{\def\PY@tc##1{\textcolor[rgb]{0.10,0.09,0.49}{##1}}}
\expandafter\def\csname PY@tok@vm\endcsname{\def\PY@tc##1{\textcolor[rgb]{0.10,0.09,0.49}{##1}}}
\expandafter\def\csname PY@tok@mh\endcsname{\def\PY@tc##1{\textcolor[rgb]{0.40,0.40,0.40}{##1}}}
\expandafter\def\csname PY@tok@cs\endcsname{\let\PY@it=\textit\def\PY@tc##1{\textcolor[rgb]{0.25,0.50,0.50}{##1}}}
\expandafter\def\csname PY@tok@ge\endcsname{\let\PY@it=\textit}
\expandafter\def\csname PY@tok@vc\endcsname{\def\PY@tc##1{\textcolor[rgb]{0.10,0.09,0.49}{##1}}}
\expandafter\def\csname PY@tok@il\endcsname{\def\PY@tc##1{\textcolor[rgb]{0.40,0.40,0.40}{##1}}}
\expandafter\def\csname PY@tok@go\endcsname{\def\PY@tc##1{\textcolor[rgb]{0.53,0.53,0.53}{##1}}}
\expandafter\def\csname PY@tok@cp\endcsname{\def\PY@tc##1{\textcolor[rgb]{0.74,0.48,0.00}{##1}}}
\expandafter\def\csname PY@tok@gi\endcsname{\def\PY@tc##1{\textcolor[rgb]{0.00,0.63,0.00}{##1}}}
\expandafter\def\csname PY@tok@gh\endcsname{\let\PY@bf=\textbf\def\PY@tc##1{\textcolor[rgb]{0.00,0.00,0.50}{##1}}}
\expandafter\def\csname PY@tok@ni\endcsname{\let\PY@bf=\textbf\def\PY@tc##1{\textcolor[rgb]{0.60,0.60,0.60}{##1}}}
\expandafter\def\csname PY@tok@nl\endcsname{\def\PY@tc##1{\textcolor[rgb]{0.63,0.63,0.00}{##1}}}
\expandafter\def\csname PY@tok@nn\endcsname{\let\PY@bf=\textbf\def\PY@tc##1{\textcolor[rgb]{0.00,0.00,1.00}{##1}}}
\expandafter\def\csname PY@tok@no\endcsname{\def\PY@tc##1{\textcolor[rgb]{0.53,0.00,0.00}{##1}}}
\expandafter\def\csname PY@tok@na\endcsname{\def\PY@tc##1{\textcolor[rgb]{0.49,0.56,0.16}{##1}}}
\expandafter\def\csname PY@tok@nb\endcsname{\def\PY@tc##1{\textcolor[rgb]{0.00,0.50,0.00}{##1}}}
\expandafter\def\csname PY@tok@nc\endcsname{\let\PY@bf=\textbf\def\PY@tc##1{\textcolor[rgb]{0.00,0.00,1.00}{##1}}}
\expandafter\def\csname PY@tok@nd\endcsname{\def\PY@tc##1{\textcolor[rgb]{0.67,0.13,1.00}{##1}}}
\expandafter\def\csname PY@tok@ne\endcsname{\let\PY@bf=\textbf\def\PY@tc##1{\textcolor[rgb]{0.82,0.25,0.23}{##1}}}
\expandafter\def\csname PY@tok@nf\endcsname{\def\PY@tc##1{\textcolor[rgb]{0.00,0.00,1.00}{##1}}}
\expandafter\def\csname PY@tok@si\endcsname{\let\PY@bf=\textbf\def\PY@tc##1{\textcolor[rgb]{0.73,0.40,0.53}{##1}}}
\expandafter\def\csname PY@tok@s2\endcsname{\def\PY@tc##1{\textcolor[rgb]{0.73,0.13,0.13}{##1}}}
\expandafter\def\csname PY@tok@nt\endcsname{\let\PY@bf=\textbf\def\PY@tc##1{\textcolor[rgb]{0.00,0.50,0.00}{##1}}}
\expandafter\def\csname PY@tok@nv\endcsname{\def\PY@tc##1{\textcolor[rgb]{0.10,0.09,0.49}{##1}}}
\expandafter\def\csname PY@tok@s1\endcsname{\def\PY@tc##1{\textcolor[rgb]{0.73,0.13,0.13}{##1}}}
\expandafter\def\csname PY@tok@dl\endcsname{\def\PY@tc##1{\textcolor[rgb]{0.73,0.13,0.13}{##1}}}
\expandafter\def\csname PY@tok@ch\endcsname{\let\PY@it=\textit\def\PY@tc##1{\textcolor[rgb]{0.25,0.50,0.50}{##1}}}
\expandafter\def\csname PY@tok@m\endcsname{\def\PY@tc##1{\textcolor[rgb]{0.40,0.40,0.40}{##1}}}
\expandafter\def\csname PY@tok@gp\endcsname{\let\PY@bf=\textbf\def\PY@tc##1{\textcolor[rgb]{0.00,0.00,0.50}{##1}}}
\expandafter\def\csname PY@tok@sh\endcsname{\def\PY@tc##1{\textcolor[rgb]{0.73,0.13,0.13}{##1}}}
\expandafter\def\csname PY@tok@ow\endcsname{\let\PY@bf=\textbf\def\PY@tc##1{\textcolor[rgb]{0.67,0.13,1.00}{##1}}}
\expandafter\def\csname PY@tok@sx\endcsname{\def\PY@tc##1{\textcolor[rgb]{0.00,0.50,0.00}{##1}}}
\expandafter\def\csname PY@tok@bp\endcsname{\def\PY@tc##1{\textcolor[rgb]{0.00,0.50,0.00}{##1}}}
\expandafter\def\csname PY@tok@c1\endcsname{\let\PY@it=\textit\def\PY@tc##1{\textcolor[rgb]{0.25,0.50,0.50}{##1}}}
\expandafter\def\csname PY@tok@fm\endcsname{\def\PY@tc##1{\textcolor[rgb]{0.00,0.00,1.00}{##1}}}
\expandafter\def\csname PY@tok@o\endcsname{\def\PY@tc##1{\textcolor[rgb]{0.40,0.40,0.40}{##1}}}
\expandafter\def\csname PY@tok@kc\endcsname{\let\PY@bf=\textbf\def\PY@tc##1{\textcolor[rgb]{0.00,0.50,0.00}{##1}}}
\expandafter\def\csname PY@tok@c\endcsname{\let\PY@it=\textit\def\PY@tc##1{\textcolor[rgb]{0.25,0.50,0.50}{##1}}}
\expandafter\def\csname PY@tok@mf\endcsname{\def\PY@tc##1{\textcolor[rgb]{0.40,0.40,0.40}{##1}}}
\expandafter\def\csname PY@tok@err\endcsname{\def\PY@bc##1{\setlength{\fboxsep}{0pt}\fcolorbox[rgb]{1.00,0.00,0.00}{1,1,1}{\strut ##1}}}
\expandafter\def\csname PY@tok@mb\endcsname{\def\PY@tc##1{\textcolor[rgb]{0.40,0.40,0.40}{##1}}}
\expandafter\def\csname PY@tok@ss\endcsname{\def\PY@tc##1{\textcolor[rgb]{0.10,0.09,0.49}{##1}}}
\expandafter\def\csname PY@tok@sr\endcsname{\def\PY@tc##1{\textcolor[rgb]{0.73,0.40,0.53}{##1}}}
\expandafter\def\csname PY@tok@mo\endcsname{\def\PY@tc##1{\textcolor[rgb]{0.40,0.40,0.40}{##1}}}
\expandafter\def\csname PY@tok@kd\endcsname{\let\PY@bf=\textbf\def\PY@tc##1{\textcolor[rgb]{0.00,0.50,0.00}{##1}}}
\expandafter\def\csname PY@tok@mi\endcsname{\def\PY@tc##1{\textcolor[rgb]{0.40,0.40,0.40}{##1}}}
\expandafter\def\csname PY@tok@kn\endcsname{\let\PY@bf=\textbf\def\PY@tc##1{\textcolor[rgb]{0.00,0.50,0.00}{##1}}}
\expandafter\def\csname PY@tok@cpf\endcsname{\let\PY@it=\textit\def\PY@tc##1{\textcolor[rgb]{0.25,0.50,0.50}{##1}}}
\expandafter\def\csname PY@tok@kr\endcsname{\let\PY@bf=\textbf\def\PY@tc##1{\textcolor[rgb]{0.00,0.50,0.00}{##1}}}
\expandafter\def\csname PY@tok@s\endcsname{\def\PY@tc##1{\textcolor[rgb]{0.73,0.13,0.13}{##1}}}
\expandafter\def\csname PY@tok@kp\endcsname{\def\PY@tc##1{\textcolor[rgb]{0.00,0.50,0.00}{##1}}}
\expandafter\def\csname PY@tok@w\endcsname{\def\PY@tc##1{\textcolor[rgb]{0.73,0.73,0.73}{##1}}}
\expandafter\def\csname PY@tok@kt\endcsname{\def\PY@tc##1{\textcolor[rgb]{0.69,0.00,0.25}{##1}}}
\expandafter\def\csname PY@tok@sc\endcsname{\def\PY@tc##1{\textcolor[rgb]{0.73,0.13,0.13}{##1}}}
\expandafter\def\csname PY@tok@sb\endcsname{\def\PY@tc##1{\textcolor[rgb]{0.73,0.13,0.13}{##1}}}
\expandafter\def\csname PY@tok@sa\endcsname{\def\PY@tc##1{\textcolor[rgb]{0.73,0.13,0.13}{##1}}}
\expandafter\def\csname PY@tok@k\endcsname{\let\PY@bf=\textbf\def\PY@tc##1{\textcolor[rgb]{0.00,0.50,0.00}{##1}}}
\expandafter\def\csname PY@tok@se\endcsname{\let\PY@bf=\textbf\def\PY@tc##1{\textcolor[rgb]{0.73,0.40,0.13}{##1}}}
\expandafter\def\csname PY@tok@sd\endcsname{\let\PY@it=\textit\def\PY@tc##1{\textcolor[rgb]{0.73,0.13,0.13}{##1}}}

\def\PYZbs{\char`\\}
\def\PYZus{\char`\_}
\def\PYZob{\char`\{}
\def\PYZcb{\char`\}}
\def\PYZca{\char`\^}
\def\PYZam{\char`\&}
\def\PYZlt{\char`\<}
\def\PYZgt{\char`\>}
\def\PYZsh{\char`\#}
\def\PYZpc{\char`\%}
\def\PYZdl{\char`\$}
\def\PYZhy{\char`\-}
\def\PYZsq{\char`\'}
\def\PYZdq{\char`\"}
\def\PYZti{\char`\~}
% for compatibility with earlier versions
\def\PYZat{@}
\def\PYZlb{[}
\def\PYZrb{]}
\makeatother


    % Exact colors from NB
    \definecolor{incolor}{rgb}{0.0, 0.0, 0.5}
    \definecolor{outcolor}{rgb}{0.545, 0.0, 0.0}




    % Prevent overflowing lines due to hard-to-break entities
    \sloppy
    % Setup hyperref package
    \hypersetup{
      breaklinks=true,  % so long urls are correctly broken across lines
      colorlinks=true,
      urlcolor=urlcolor,
      linkcolor=linkcolor,
      citecolor=citecolor,
      }
    % Slightly bigger margins than the latex defaults

    \usepackage{fullpage}


    \begin{document}

    \maketitle


    \hypertarget{modeling-the-future-2018}{%
\section{Modeling the Future 2018}\label{modeling-the-future-2018}}

 \begin{Verbatim}[commandchars=\\\{\}]
{\color{incolor}In [{\color{incolor}1}]:} \PY{k+kn}{import} \PY{n+nn}{numpy} \PY{k}{as} \PY{n+nn}{np}
        \PY{k+kn}{import} \PY{n+nn}{pandas} \PY{k}{as} \PY{n+nn}{pd}
        \PY{k+kn}{import} \PY{n+nn}{scipy}
        \PY{k+kn}{import} \PY{n+nn}{scipy}\PY{n+nn}{.}\PY{n+nn}{stats}
        \PY{k+kn}{import} \PY{n+nn}{matplotlib}\PY{n+nn}{.}\PY{n+nn}{pyplot} \PY{k}{as} \PY{n+nn}{plt}
        \PY{k+kn}{import} \PY{n+nn}{seaborn} \PY{k}{as} \PY{n+nn}{sns}
        \PY{k+kn}{import} \PY{n+nn}{statsmodels}\PY{n+nn}{.}\PY{n+nn}{api} \PY{k}{as} \PY{n+nn}{sm}
        \PY{k+kn}{import} \PY{n+nn}{statsmodels}\PY{n+nn}{.}\PY{n+nn}{formula}\PY{n+nn}{.}\PY{n+nn}{api} \PY{k}{as} \PY{n+nn}{smf}
        \PY{n}{sns}\PY{o}{.}\PY{n}{set}\PY{p}{(}\PY{p}{)}
        \PY{k+kn}{import} \PY{n+nn}{warnings}
        \PY{n}{warnings}\PY{o}{.}\PY{n}{simplefilter}\PY{p}{(}\PY{l+s+s1}{\PYZsq{}}\PY{l+s+s1}{ignore}\PY{l+s+s1}{\PYZsq{}}\PY{p}{,}\PY{n+ne}{FutureWarning}\PY{p}{)}
\end{Verbatim}

    \hypertarget{scenario-response-topic-1--automobile-insurance}{%
\subsection{Scenario Response Topic 1- Automobile
Insurance}\label{scenario-response-topic-1--automobile-insurance}}

Answers are in \emph{italics}.

 \begin{Verbatim}[commandchars=\\\{\}]
{\color{incolor}In [{\color{incolor}2}]:} \PY{n}{insurance} \PY{o}{=} \PY{n}{pd}\PY{o}{.}\PY{n}{read\PYZus{}excel}\PY{p}{(}\PY{l+s+s1}{\PYZsq{}}\PY{l+s+s1}{Scenario\PYZhy{}Response\PYZhy{}1\PYZhy{}Auto\PYZhy{}Insurance\PYZhy{}Data.xlsx}\PY{l+s+s1}{\PYZsq{}}\PY{p}{)}
\end{Verbatim}

    \fullline

    \textbf{Question 1:} Estimate the probability that a Tonga owner has a
loss during one year

\emph{\textasciitilde{}9.87\%}


    \fullline
 \begin{Verbatim}[commandchars=\\\{\}]
{\color{incolor}In [{\color{incolor}3}]:} \PY{n}{tongas} \PY{o}{=} \PY{n}{insurance}\PY{o}{.}\PY{n}{groupby}\PY{p}{(}\PY{l+s+s1}{\PYZsq{}}\PY{l+s+s1}{Car Brand}\PY{l+s+s1}{\PYZsq{}}\PY{p}{)}\PY{o}{.}\PY{n}{get\PYZus{}group}\PY{p}{(}\PY{l+s+s1}{\PYZsq{}}\PY{l+s+s1}{Tonga}\PY{l+s+s1}{\PYZsq{}}\PY{p}{)}
        \PY{n}{tongas\PYZus{}percent\PYZus{}loss} \PY{o}{=} \PY{l+m+mi}{1} \PY{o}{\PYZhy{}} \PY{p}{(}\PY{n}{tongas}\PY{o}{.}\PY{n}{groupby}\PY{p}{(}\PY{l+s+s1}{\PYZsq{}}\PY{l+s+s1}{Loss}\PY{l+s+s1}{\PYZsq{}}\PY{p}{)}\PY{o}{.}\PY{n}{get\PYZus{}group}\PY{p}{(}\PY{l+m+mi}{0}\PY{p}{)}\PY{o}{.}\PY{n}{shape}\PY{p}{[}\PY{l+m+mi}{0}\PY{p}{]} \PY{o}{/} \PY{n}{tongas}\PY{o}{.}\PY{n}{shape}\PY{p}{[}\PY{l+m+mi}{0}\PY{p}{]}\PY{p}{)}
        \PY{n}{tongas\PYZus{}percent\PYZus{}loss}
\end{Verbatim}

\begin{Verbatim}[commandchars=\\\{\}]
{\color{outcolor}Out[{\color{outcolor}3}]:} 0.09872102318145481
\end{Verbatim}
        \fullline


    \textbf{Question 2:} Estimate the loss on a policy, given that the
policyholder owned a Bergen.


\emph{\textasciitilde{}\$1100.36}

        \fullline

 \begin{Verbatim}[commandchars=\\\{\}]
{\color{incolor}In [{\color{incolor}4}]:} \PY{n}{insurance}\PY{o}{.}\PY{n}{groupby}\PY{p}{(}\PY{l+s+s1}{\PYZsq{}}\PY{l+s+s1}{Car Brand}\PY{l+s+s1}{\PYZsq{}}\PY{p}{)}\PY{o}{.}\PY{n}{get\PYZus{}group}\PY{p}{(}\PY{l+s+s1}{\PYZsq{}}\PY{l+s+s1}{Bergen}\PY{l+s+s1}{\PYZsq{}}\PY{p}{)}\PY{p}{[}\PY{l+s+s1}{\PYZsq{}}\PY{l+s+s1}{Loss}\PY{l+s+s1}{\PYZsq{}}\PY{p}{]}\PY{o}{.}\PY{n}{mean}\PY{p}{(}\PY{p}{)}
\end{Verbatim}

 \begin{Verbatim}[commandchars=\\\{\}]
{\color{outcolor}Out[{\color{outcolor}4}]:} 1100.356084319945
\end{Verbatim}


        \fullline


    \textbf{Question 3:} Do you see a relationship between Age and Loss,
given a loss has occurred? If so, what type of relationship? What could
be a logical reason for this trend?

\emph{Yes. Given that loss has occured, young people lose on average
15\% more money as old people. A logical reason for this trend is that
young people are more careless and are less experienced drivers,
therefore the accidents that they get in are worse.}


        \fullline

 \begin{Verbatim}[commandchars=\\\{\}]
{\color{incolor}In [{\color{incolor}5}]:} \PY{n}{age\PYZus{}mean} \PY{o}{=} \PY{n}{insurance}\PY{p}{[}\PY{n}{insurance}\PY{o}{.}\PY{n}{Loss} \PY{o}{\PYZgt{}} \PY{l+m+mi}{0}\PY{p}{]}\PY{o}{.}\PY{n}{groupby}\PY{p}{(}\PY{l+s+s1}{\PYZsq{}}\PY{l+s+s1}{Age}\PY{l+s+s1}{\PYZsq{}}\PY{p}{)}\PY{o}{.}\PY{n}{mean}\PY{p}{(}\PY{p}{)}
        \PY{n}{sns}\PY{o}{.}\PY{n}{barplot}\PY{p}{(}\PY{n}{x}\PY{o}{=}\PY{n}{age\PYZus{}mean}\PY{o}{.}\PY{n}{index}\PY{p}{,} \PY{n}{y}\PY{o}{=}\PY{l+s+s1}{\PYZsq{}}\PY{l+s+s1}{Loss}\PY{l+s+s1}{\PYZsq{}}\PY{p}{,} \PY{n}{data}\PY{o}{=}\PY{n}{age\PYZus{}mean}\PY{p}{)}\PY{o}{.}\PY{n}{set\PYZus{}title}\PY{p}{(}\PY{l+s+s1}{\PYZsq{}}\PY{l+s+s1}{Mean Loss per Age Group}\PY{l+s+s1}{\PYZsq{}}\PY{p}{)}
\end{Verbatim}

 \begin{Verbatim}[commandchars=\\\{\}]
{\color{outcolor}Out[{\color{outcolor}5}]:} Text(0.5,1,'Mean Loss per Age Group')
\end{Verbatim}

    \begin{center}
    \adjustimage{max size={0.9\linewidth}{0.9\paperheight}}{Scenario Response Topic 1- Automobile Insurance_files/Scenario Response Topic 1- Automobile Insurance_10_1.png}
    \end{center}
    { \hspace*{\fill} \\}

 \begin{Verbatim}[commandchars=\\\{\}]
{\color{incolor}In [{\color{incolor}6}]:} \PY{n}{losses} \PY{o}{=} \PY{n}{insurance}\PY{p}{[}\PY{n}{insurance}\PY{o}{.}\PY{n}{Loss} \PY{o}{\PYZgt{}} \PY{l+m+mi}{0}\PY{p}{]}
        \PY{n}{smf}\PY{o}{.}\PY{n}{ols}\PY{p}{(}\PY{l+s+s1}{\PYZsq{}}\PY{l+s+s1}{Loss \PYZti{} Age \PYZhy{} 1}\PY{l+s+s1}{\PYZsq{}}\PY{p}{,} \PY{n}{data}\PY{o}{=}\PY{n}{losses}\PY{p}{)}\PY{o}{.}\PY{n}{fit}\PY{p}{(}\PY{p}{)}\PY{o}{.}\PY{n}{summary}\PY{p}{(}\PY{p}{)}
\end{Verbatim}

 \begin{Verbatim}[commandchars=\\\{\}]
{\color{outcolor}Out[{\color{outcolor}6}]:} <class 'statsmodels.iolib.summary.Summary'>
        """
                                    OLS Regression Results
        ==============================================================================
        Dep. Variable:                   Loss   R-squared:                       0.019
        Model:                            OLS   Adj. R-squared:                  0.017
        Method:                 Least Squares   F-statistic:                     12.68
        Date:                Thu, 06 Dec 2018   Prob (F-statistic):           3.49e-06
        Time:                        10:26:41   Log-Likelihood:                -13956.
        No. Observations:                1347   AIC:                         2.792e+04
        Df Residuals:                    1344   BIC:                         2.793e+04
        Df Model:                           2
        Covariance Type:            nonrobust
        ==============================================================================
                         coef    std err          t      P>|t|      [0.025      0.975]
        ------------------------------------------------------------------------------
        Age[M]      6121.2834    392.646     15.590      0.000    5351.018    6891.549
        Age[O]      7425.7139    368.257     20.164      0.000    6703.293    8148.135
        Age[Y]      8697.2409    330.915     26.282      0.000    8048.076    9346.406
        ==============================================================================
        Omnibus:                      634.540   Durbin-Watson:                   2.033
        Prob(Omnibus):                  0.000   Jarque-Bera (JB):             3790.198
        Skew:                           2.141   Prob(JB):                         0.00
        Kurtosis:                      10.013   Cond. No.                         1.19
        ==============================================================================
        """
\end{Verbatim}


\newpage
    \fullline


    \textbf{Question 4:} Do you see a relationship between gender and the
frequency of a loss occurring?

\emph{Males are \textasciitilde{}2.2\% more likely to have a loss (low
p-value so this is statistically significant).}


    \fullline
 \begin{Verbatim}[commandchars=\\\{\}]
{\color{incolor}In [{\color{incolor}7}]:} \PY{n}{males} \PY{o}{=} \PY{n}{insurance}\PY{o}{.}\PY{n}{groupby}\PY{p}{(}\PY{l+s+s1}{\PYZsq{}}\PY{l+s+s1}{Gender}\PY{l+s+s1}{\PYZsq{}}\PY{p}{)}\PY{o}{.}\PY{n}{get\PYZus{}group}\PY{p}{(}\PY{l+s+s1}{\PYZsq{}}\PY{l+s+s1}{M}\PY{l+s+s1}{\PYZsq{}}\PY{p}{)}
        \PY{n}{females} \PY{o}{=} \PY{n}{insurance}\PY{o}{.}\PY{n}{groupby}\PY{p}{(}\PY{l+s+s1}{\PYZsq{}}\PY{l+s+s1}{Gender}\PY{l+s+s1}{\PYZsq{}}\PY{p}{)}\PY{o}{.}\PY{n}{get\PYZus{}group}\PY{p}{(}\PY{l+s+s1}{\PYZsq{}}\PY{l+s+s1}{F}\PY{l+s+s1}{\PYZsq{}}\PY{p}{)}
        \PY{n}{gender\PYZus{}lossfreq} \PY{o}{=} \PY{n}{pd}\PY{o}{.}\PY{n}{DataFrame}\PY{p}{(}\PY{p}{\PYZob{}}\PY{l+s+s1}{\PYZsq{}}\PY{l+s+s1}{Gender}\PY{l+s+s1}{\PYZsq{}} \PY{p}{:} \PY{p}{[}\PY{l+s+s1}{\PYZsq{}}\PY{l+s+s1}{M}\PY{l+s+s1}{\PYZsq{}}\PY{p}{,} \PY{l+s+s1}{\PYZsq{}}\PY{l+s+s1}{F}\PY{l+s+s1}{\PYZsq{}}\PY{p}{]}\PY{p}{,}
                                        \PY{l+s+s1}{\PYZsq{}}\PY{l+s+s1}{Loss Frequency}\PY{l+s+s1}{\PYZsq{}} \PY{p}{:} \PY{p}{[}\PY{n}{males}\PY{p}{[}\PY{n}{males}\PY{o}{.}\PY{n}{Loss} \PY{o}{!=} \PY{l+m+mi}{0}\PY{p}{]}\PY{o}{.}\PY{n}{size} \PY{o}{/} \PY{n}{males}\PY{o}{.}\PY{n}{size}\PY{p}{,}
                                                            \PY{n}{females}\PY{p}{[}\PY{n}{females}\PY{o}{.}\PY{n}{Loss} \PY{o}{!=} \PY{l+m+mi}{0}\PY{p}{]}\PY{o}{.}\PY{n}{size} \PY{o}{/} \PY{n}{females}\PY{o}{.}\PY{n}{size}\PY{p}{]}\PY{p}{\PYZcb{}}\PY{p}{)}
        \PY{n}{sns}\PY{o}{.}\PY{n}{barplot}\PY{p}{(}\PY{n}{x}\PY{o}{=}\PY{l+s+s1}{\PYZsq{}}\PY{l+s+s1}{Gender}\PY{l+s+s1}{\PYZsq{}}\PY{p}{,} \PY{n}{y}\PY{o}{=}\PY{l+s+s1}{\PYZsq{}}\PY{l+s+s1}{Loss Frequency}\PY{l+s+s1}{\PYZsq{}}\PY{p}{,} \PY{n}{data}\PY{o}{=}\PY{n}{gender\PYZus{}lossfreq}\PY{p}{)}\PY{o}{.}\PY{n}{set\PYZus{}title}\PY{p}{(}\PY{l+s+s1}{\PYZsq{}}\PY{l+s+s1}{Loss Frequency vs Gender}\PY{l+s+s1}{\PYZsq{}}\PY{p}{)}
\end{Verbatim}

 \begin{Verbatim}[commandchars=\\\{\}]
{\color{outcolor}Out[{\color{outcolor}7}]:} Text(0.5,1,'Loss Frequency vs Gender')
\end{Verbatim}

    \begin{center}
    \adjustimage{max size={0.9\linewidth}{0.9\paperheight}}{Scenario Response Topic 1- Automobile Insurance_files/Scenario Response Topic 1- Automobile Insurance_13_1.png}
    \end{center}
    { \hspace*{\fill} \\}

 \begin{Verbatim}[commandchars=\\\{\}]
{\color{incolor}In [{\color{incolor}8}]:} \PY{n+nb}{print}\PY{p}{(}\PY{l+s+s1}{\PYZsq{}}\PY{l+s+s1}{\PYZsh{} of Male losses: }\PY{l+s+s1}{\PYZsq{}} \PY{o}{+} \PY{n+nb}{str}\PY{p}{(}\PY{n}{males}\PY{p}{[}\PY{n}{males}\PY{o}{.}\PY{n}{Loss} \PY{o}{!=} \PY{l+m+mi}{0}\PY{p}{]}\PY{o}{.}\PY{n}{shape}\PY{p}{[}\PY{l+m+mi}{0}\PY{p}{]}\PY{p}{)}\PY{p}{)}
        \PY{n+nb}{print}\PY{p}{(}\PY{l+s+s1}{\PYZsq{}}\PY{l+s+s1}{\PYZsh{} of Female losses: }\PY{l+s+s1}{\PYZsq{}} \PY{o}{+} \PY{n+nb}{str}\PY{p}{(}\PY{n}{females}\PY{p}{[}\PY{n}{females}\PY{o}{.}\PY{n}{Loss} \PY{o}{!=} \PY{l+m+mi}{0}\PY{p}{]}\PY{o}{.}\PY{n}{shape}\PY{p}{[}\PY{l+m+mi}{0}\PY{p}{]}\PY{p}{)}\PY{p}{)}
\end{Verbatim}

 \begin{Verbatim}[commandchars=\\\{\}]
\# of Male losses: 728
\# of Female losses: 619

    \end{Verbatim}

 \begin{Verbatim}[commandchars=\\\{\}]
{\color{incolor}In [{\color{incolor}9}]:} \PY{n}{insurance}\PY{o}{.}\PY{n}{groupby}\PY{p}{(}\PY{l+s+s1}{\PYZsq{}}\PY{l+s+s1}{Gender}\PY{l+s+s1}{\PYZsq{}}\PY{p}{)}\PY{o}{.}\PY{n}{mean}\PY{p}{(}\PY{p}{)}\PY{o}{.}\PY{n}{iloc}\PY{p}{[}\PY{p}{:}\PY{p}{,} \PY{l+m+mi}{1}\PY{p}{:}\PY{l+m+mi}{2}\PY{p}{]}
\end{Verbatim}


\begin{Verbatim}[commandchars=\\\{\}]
{\color{outcolor}Out[{\color{outcolor}9}]:}                Loss
        Gender
        F        932.104907
        M       1103.670467
\end{Verbatim}

 \begin{Verbatim}[commandchars=\\\{\}]
{\color{incolor}In [{\color{incolor}10}]:} \PY{n}{loss\PYZus{}transform} \PY{o}{=} \PY{n}{np}\PY{o}{.}\PY{n}{vectorize}\PY{p}{(}\PY{n+nb}{min}\PY{p}{)}
\end{Verbatim}

 \begin{Verbatim}[commandchars=\\\{\}]
{\color{incolor}In [{\color{incolor}11}]:} \PY{n}{insurance4} \PY{o}{=} \PY{n}{insurance}\PY{o}{.}\PY{n}{copy}\PY{p}{(}\PY{p}{)}
         \PY{n}{insurance4}\PY{p}{[}\PY{l+s+s1}{\PYZsq{}}\PY{l+s+s1}{HasLoss}\PY{l+s+s1}{\PYZsq{}}\PY{p}{]} \PY{o}{=} \PY{n}{loss\PYZus{}transform}\PY{p}{(}\PY{n}{insurance4}\PY{p}{[}\PY{l+s+s1}{\PYZsq{}}\PY{l+s+s1}{Loss}\PY{l+s+s1}{\PYZsq{}}\PY{p}{]}\PY{p}{,} \PY{l+m+mi}{1}\PY{p}{)}
         \PY{n}{smf}\PY{o}{.}\PY{n}{ols}\PY{p}{(}\PY{l+s+s1}{\PYZsq{}}\PY{l+s+s1}{HasLoss \PYZti{} Gender \PYZhy{} 1}\PY{l+s+s1}{\PYZsq{}}\PY{p}{,} \PY{n}{data}\PY{o}{=}\PY{n}{insurance4}\PY{p}{)}\PY{o}{.}\PY{n}{fit}\PY{p}{(}\PY{p}{)}\PY{o}{.}\PY{n}{summary}\PY{p}{(}\PY{p}{)}
\end{Verbatim}

 \begin{Verbatim}[commandchars=\\\{\}]
{\color{outcolor}Out[{\color{outcolor}11}]:} <class 'statsmodels.iolib.summary.Summary'>
         """
                                     OLS Regression Results
         ==============================================================================
         Dep. Variable:                HasLoss   R-squared:                       0.001
         Model:                            OLS   Adj. R-squared:                  0.001
         Method:                 Least Squares   F-statistic:                     10.20
         Date:                Thu, 06 Dec 2018   Prob (F-statistic):            0.00141
         Time:                        10:26:42   Log-Likelihood:                -3436.7
         No. Observations:               10008   AIC:                             6877.
         Df Residuals:                   10006   BIC:                             6892.
         Df Model:                           1
         Covariance Type:            nonrobust
         ==============================================================================
                          coef    std err          t      P>|t|      [0.025      0.975]
         ------------------------------------------------------------------------------
         Gender[F]      0.1237      0.005     25.650      0.000       0.114       0.133
         Gender[M]      0.1455      0.005     30.167      0.000       0.136       0.155
         ==============================================================================
         Omnibus:                     3796.746   Durbin-Watson:                   1.965
         Prob(Omnibus):                  0.000   Jarque-Bera (JB):            10400.902
         Skew:                           2.138   Prob(JB):                         0.00
         Kurtosis:                       5.580   Cond. No.                         1.00
         ==============================================================================

         """
\end{Verbatim}
        \fullline


    \textbf{Question 5:} Which car brand and gender combination has the
least expected payout?


\emph{Female with a Davis (lowest mean payout, payout is loss given that
loss has occured)}


        \fullline
 \begin{Verbatim}[commandchars=\\\{\}]
{\color{incolor}In [{\color{incolor}12}]:} \PY{n}{males}\PY{p}{[}\PY{n}{males}\PY{o}{.}\PY{n}{Loss} \PY{o}{\PYZgt{}} \PY{l+m+mi}{0}\PY{p}{]}\PY{o}{.}\PY{n}{groupby}\PY{p}{(}\PY{l+s+s1}{\PYZsq{}}\PY{l+s+s1}{Car Brand}\PY{l+s+s1}{\PYZsq{}}\PY{p}{)}\PY{o}{.}\PY{n}{mean}\PY{p}{(}\PY{p}{)}\PY{o}{.}\PY{n}{iloc}\PY{p}{[}\PY{p}{:}\PY{p}{,} \PY{l+m+mi}{1}\PY{p}{:}\PY{l+m+mi}{2}\PY{p}{]}
\end{Verbatim}

 \begin{Verbatim}[commandchars=\\\{\}]
{\color{outcolor}Out[{\color{outcolor}12}]:}                   Loss
         Car Brand
         Bergen     7646.016782
         Davis      6780.126582
         Montes     7736.295132
         Tonga      8372.143042
\end{Verbatim}

 \begin{Verbatim}[commandchars=\\\{\}]
{\color{incolor}In [{\color{incolor}13}]:} \PY{n}{females}\PY{p}{[}\PY{n}{females}\PY{o}{.}\PY{n}{Loss} \PY{o}{\PYZgt{}} \PY{l+m+mi}{0}\PY{p}{]}\PY{o}{.}\PY{n}{groupby}\PY{p}{(}\PY{l+s+s1}{\PYZsq{}}\PY{l+s+s1}{Car Brand}\PY{l+s+s1}{\PYZsq{}}\PY{p}{)}\PY{o}{.}\PY{n}{mean}\PY{p}{(}\PY{p}{)}\PY{o}{.}\PY{n}{iloc}\PY{p}{[}\PY{p}{:}\PY{p}{,} \PY{l+m+mi}{1}\PY{p}{:}\PY{l+m+mi}{2}\PY{p}{]}
\end{Verbatim}

 \begin{Verbatim}[commandchars=\\\{\}]
{\color{outcolor}Out[{\color{outcolor}13}]:}                   Loss
         Car Brand
         Bergen     7882.327546
         Davis      6737.887209
         Montes     7209.681824
         Tonga      8656.186828
\end{Verbatim}
        \fullline


    \textbf{Question 6:} What type of distribution does a Davis
policyholder's loss have, given that there is a loss? Graph the
distribution and comment on its shape.


        \fullline
 \begin{Verbatim}[commandchars=\\\{\}]
{\color{incolor}In [{\color{incolor}14}]:} \PY{n}{daviss} \PY{o}{=} \PY{n}{insurance}\PY{o}{.}\PY{n}{groupby}\PY{p}{(}\PY{l+s+s1}{\PYZsq{}}\PY{l+s+s1}{Car Brand}\PY{l+s+s1}{\PYZsq{}}\PY{p}{)}\PY{o}{.}\PY{n}{get\PYZus{}group}\PY{p}{(}\PY{l+s+s1}{\PYZsq{}}\PY{l+s+s1}{Davis}\PY{l+s+s1}{\PYZsq{}}\PY{p}{)}
         \PY{n}{sns}\PY{o}{.}\PY{n}{distplot}\PY{p}{(}\PY{n}{daviss}\PY{p}{[}\PY{n}{daviss}\PY{o}{.}\PY{n}{Loss} \PY{o}{\PYZgt{}} \PY{l+m+mi}{0}\PY{p}{]}\PY{p}{[}\PY{l+s+s1}{\PYZsq{}}\PY{l+s+s1}{Loss}\PY{l+s+s1}{\PYZsq{}}\PY{p}{]}\PY{p}{)}\PY{o}{.}\PY{n}{set\PYZus{}title}\PY{p}{(}\PY{l+s+s1}{\PYZsq{}}\PY{l+s+s1}{Probability Distribution of Losses on Davis Policies}\PY{l+s+s1}{\PYZsq{}}\PY{p}{)}
\end{Verbatim}

 \begin{Verbatim}[commandchars=\\\{\}]
{\color{outcolor}Out[{\color{outcolor}14}]:} Text(0.5,1,'Probability Distribution of Losses on Davis Policies')
\end{Verbatim}

    \begin{center}
    \adjustimage{max size={0.9\linewidth}{0.9\paperheight}}{Scenario Response Topic 1- Automobile Insurance_files/Scenario Response Topic 1- Automobile Insurance_22_1.png}
    \end{center}

    \emph{This looks most like an exponential distribution:}

    \begin{center}
    \adjustimage{max size={0.8\linewidth}{0.8\paperheight}}{Scenario Response Topic 1- Automobile Insurance_files/Exponential.png}
    \end{center}

\newpage
        \fullline


    \textbf{Question 7:} A family owns two cars, a Davis and a Tonga. What
is the probability that exactly one of the policies will have a loss in
the next year? The parents (middle-aged) drive the Tonga, while the
children (young) drive the Davis.


    \fullline
\emph{\textasciitilde{}21.32\%}

 \begin{Verbatim}[commandchars=\\\{\}]
{\color{incolor}In [{\color{incolor}15}]:} \PY{n}{young\PYZus{}davis\PYZus{}percent\PYZus{}loss} \PY{o}{=} \PY{n+nb}{len}\PY{p}{(}\PY{n}{daviss}\PY{p}{[}\PY{p}{(}\PY{n}{daviss}\PY{o}{.}\PY{n}{Loss} \PY{o}{\PYZgt{}} \PY{l+m+mi}{0}\PY{p}{)} \PY{o}{\PYZam{}} \PY{p}{(}\PY{n}{daviss}\PY{o}{.}\PY{n}{Age} \PY{o}{==} \PY{l+s+s1}{\PYZsq{}}\PY{l+s+s1}{Y}\PY{l+s+s1}{\PYZsq{}}\PY{p}{)}\PY{p}{]}\PY{p}{)} \PY{o}{/} \PY{n+nb}{len}\PY{p}{(}\PY{n}{daviss}\PY{p}{[}\PY{n}{daviss}\PY{o}{.}\PY{n}{Age} \PY{o}{==} \PY{l+s+s1}{\PYZsq{}}\PY{l+s+s1}{Y}\PY{l+s+s1}{\PYZsq{}}\PY{p}{]}\PY{p}{)}
         \PY{n}{mid\PYZus{}tongas\PYZus{}percent\PYZus{}loss} \PY{o}{=} \PY{n+nb}{len}\PY{p}{(}\PY{n}{tongas}\PY{p}{[}\PY{p}{(}\PY{n}{tongas}\PY{o}{.}\PY{n}{Loss} \PY{o}{\PYZgt{}} \PY{l+m+mi}{0}\PY{p}{)} \PY{o}{\PYZam{}} \PY{p}{(}\PY{n}{tongas}\PY{o}{.}\PY{n}{Age} \PY{o}{==} \PY{l+s+s1}{\PYZsq{}}\PY{l+s+s1}{M}\PY{l+s+s1}{\PYZsq{}}\PY{p}{)}\PY{p}{]}\PY{p}{)} \PY{o}{/} \PY{n+nb}{len}\PY{p}{(}\PY{n}{tongas}\PY{p}{[}\PY{n}{tongas}\PY{o}{.}\PY{n}{Age} \PY{o}{==} \PY{l+s+s1}{\PYZsq{}}\PY{l+s+s1}{Y}\PY{l+s+s1}{\PYZsq{}}\PY{p}{]}\PY{p}{)}
\end{Verbatim}

 \begin{Verbatim}[commandchars=\\\{\}]
{\color{incolor}In [{\color{incolor}16}]:} \PY{p}{(}\PY{n}{young\PYZus{}davis\PYZus{}percent\PYZus{}loss} \PY{o}{*} \PY{p}{(}\PY{l+m+mi}{1} \PY{o}{\PYZhy{}} \PY{n}{mid\PYZus{}tongas\PYZus{}percent\PYZus{}loss}\PY{p}{)}\PY{p}{)} \PY{o}{+} \PY{p}{(}\PY{p}{(}\PY{l+m+mi}{1} \PY{o}{\PYZhy{}} \PY{n}{young\PYZus{}davis\PYZus{}percent\PYZus{}loss}\PY{p}{)} \PY{o}{*} \PY{n}{mid\PYZus{}tongas\PYZus{}percent\PYZus{}loss}\PY{p}{)}
\end{Verbatim}

 \begin{Verbatim}[commandchars=\\\{\}]
{\color{outcolor}Out[{\color{outcolor}16}]:} 0.21317909701016163
\end{Verbatim}
        \fullline


    \textbf{Question 8:} Joe, a 20-year old male is looking to buy insurance
for a new Davis. His quote is \$1800 per year. Joe comes to you, and
asks you to analyze the data to see if the quote is higher than the
expected amount that he would lose each year without insurance. What is
the expected loss per year? What additional value does an insurance
policy provide, if the expected loss is lower than the quote?

\emph{Expected Loss Per Year: \$1387.91}

\emph{If the expected loss is lower than the quote, this value is the
expected amount a policy holder will have to pay on top of their
deductible (quote) as the insurance company does not pay anything if
this is true.}


        \fullline
 \begin{Verbatim}[commandchars=\\\{\}]
{\color{incolor}In [{\color{incolor}17}]:} \PY{n}{insurance}\PY{p}{[}\PY{p}{(}\PY{n}{insurance}\PY{p}{[}\PY{l+s+s1}{\PYZsq{}}\PY{l+s+s1}{Car Brand}\PY{l+s+s1}{\PYZsq{}}\PY{p}{]} \PY{o}{==} \PY{l+s+s1}{\PYZsq{}}\PY{l+s+s1}{Davis}\PY{l+s+s1}{\PYZsq{}}\PY{p}{)} \PY{o}{\PYZam{}} \PY{p}{(}\PY{n}{insurance}\PY{p}{[}\PY{l+s+s1}{\PYZsq{}}\PY{l+s+s1}{Gender}\PY{l+s+s1}{\PYZsq{}}\PY{p}{]} \PY{o}{==} \PY{l+s+s1}{\PYZsq{}}\PY{l+s+s1}{M}\PY{l+s+s1}{\PYZsq{}}\PY{p}{)} \PY{o}{\PYZam{}} \PY{p}{(}\PY{n}{insurance}\PY{p}{[}\PY{l+s+s1}{\PYZsq{}}\PY{l+s+s1}{Age}\PY{l+s+s1}{\PYZsq{}}\PY{p}{]} \PY{o}{==} \PY{l+s+s1}{\PYZsq{}}\PY{l+s+s1}{Y}\PY{l+s+s1}{\PYZsq{}}\PY{p}{)}\PY{p}{]}\PY{p}{[}\PY{l+s+s1}{\PYZsq{}}\PY{l+s+s1}{Loss}\PY{l+s+s1}{\PYZsq{}}\PY{p}{]}\PY{o}{.}\PY{n}{mean}\PY{p}{(}\PY{p}{)}
\end{Verbatim}

 \begin{Verbatim}[commandchars=\\\{\}]
{\color{outcolor}Out[{\color{outcolor}17}]:} 1387.9104881888504
\end{Verbatim}
        \fullline


    \textbf{Question 9:} Calculate the pure premium for a young male driving
a Davis, using a profit margin of 15\% and fixed costs of \$134 per
policy holder. If we ignore interest, pure premium of an insurable event
equals the likelihood the event will happen (called the frequency)
multiplied by the insurer's expected cost when the event happens (called
the severity). Interpret this number.

\emph{The pure premium for a young male driving a Davis is about
\$448.82 if the company expects to make 15\% profit on the pure premium
and a \textbackslash{}\$134 fee is added on top of the pure premium
amount.}

        \fullline
 \begin{Verbatim}[commandchars=\\\{\}]
{\color{incolor}In [{\color{incolor}23}]:} \PY{k}{def} \PY{n+nf}{get\PYZus{}mean\PYZus{}loss}\PY{p}{(}\PY{n}{brand}\PY{p}{,} \PY{n}{gender}\PY{p}{,} \PY{n}{age}\PY{p}{)}\PY{p}{:}
             \PY{k}{return} \PY{n}{insurance}\PY{p}{[}\PY{p}{(}\PY{n}{insurance}\PY{p}{[}\PY{l+s+s1}{\PYZsq{}}\PY{l+s+s1}{Car Brand}\PY{l+s+s1}{\PYZsq{}}\PY{p}{]} \PY{o}{==} \PY{n}{brand}\PY{p}{)} \PY{o}{\PYZam{}} \PY{p}{(}\PY{n}{insurance}\PY{p}{[}\PY{l+s+s1}{\PYZsq{}}\PY{l+s+s1}{Gender}\PY{l+s+s1}{\PYZsq{}}\PY{p}{]} \PY{o}{==} \PY{n}{gender}\PY{p}{)} \PY{o}{\PYZam{}} \PY{p}{(}\PY{n}{insurance}\PY{p}{[}\PY{l+s+s1}{\PYZsq{}}\PY{l+s+s1}{Age}\PY{l+s+s1}{\PYZsq{}}\PY{p}{]} \PY{o}{==} \PY{n}{age}\PY{p}{)}\PY{p}{]}\PY{p}{[}\PY{l+s+s1}{\PYZsq{}}\PY{l+s+s1}{Loss}\PY{l+s+s1}{\PYZsq{}}\PY{p}{]}\PY{o}{.}\PY{n}{mean}\PY{p}{(}\PY{p}{)}
         \PY{k}{def} \PY{n+nf}{get\PYZus{}loss\PYZus{}freq}\PY{p}{(}\PY{n}{brand}\PY{p}{,} \PY{n}{gender}\PY{p}{,} \PY{n}{age}\PY{p}{)}\PY{p}{:}
             \PY{n}{total} \PY{o}{=} \PY{n}{insurance}\PY{p}{[}\PY{p}{(}\PY{n}{insurance}\PY{p}{[}\PY{l+s+s1}{\PYZsq{}}\PY{l+s+s1}{Car Brand}\PY{l+s+s1}{\PYZsq{}}\PY{p}{]} \PY{o}{==} \PY{n}{brand}\PY{p}{)} \PY{o}{\PYZam{}} \PY{p}{(}\PY{n}{insurance}\PY{p}{[}\PY{l+s+s1}{\PYZsq{}}\PY{l+s+s1}{Gender}\PY{l+s+s1}{\PYZsq{}}\PY{p}{]} \PY{o}{==} \PY{n}{gender}\PY{p}{)} \PY{o}{\PYZam{}} \PY{p}{(}\PY{n}{insurance}\PY{p}{[}\PY{l+s+s1}{\PYZsq{}}\PY{l+s+s1}{Age}\PY{l+s+s1}{\PYZsq{}}\PY{p}{]} \PY{o}{==} \PY{n}{age}\PY{p}{)}\PY{p}{]}
             \PY{n}{loss} \PY{o}{=} \PY{n}{total}\PY{p}{[}\PY{n}{total}\PY{o}{.}\PY{n}{Loss} \PY{o}{\PYZgt{}} \PY{l+m+mi}{0}\PY{p}{]}
             \PY{k}{return} \PY{n+nb}{len}\PY{p}{(}\PY{n}{loss}\PY{p}{)} \PY{o}{/} \PY{n+nb}{len}\PY{p}{(}\PY{n}{total}\PY{p}{)}
\end{Verbatim}

 \begin{Verbatim}[commandchars=\\\{\}]
{\color{incolor}In [{\color{incolor}19}]:} \PY{n}{pure\PYZus{}premium} \PY{o}{=} \PY{n}{get\PYZus{}loss\PYZus{}freq}\PY{p}{(}\PY{l+s+s1}{\PYZsq{}}\PY{l+s+s1}{Davis}\PY{l+s+s1}{\PYZsq{}}\PY{p}{,} \PY{l+s+s1}{\PYZsq{}}\PY{l+s+s1}{M}\PY{l+s+s1}{\PYZsq{}}\PY{p}{,} \PY{l+s+s1}{\PYZsq{}}\PY{l+s+s1}{Y}\PY{l+s+s1}{\PYZsq{}}\PY{p}{)} \PY{o}{*} \PY{n}{get\PYZus{}mean\PYZus{}loss}\PY{p}{(}\PY{l+s+s1}{\PYZsq{}}\PY{l+s+s1}{Davis}\PY{l+s+s1}{\PYZsq{}}\PY{p}{,} \PY{l+s+s1}{\PYZsq{}}\PY{l+s+s1}{M}\PY{l+s+s1}{\PYZsq{}}\PY{p}{,} \PY{l+s+s1}{\PYZsq{}}\PY{l+s+s1}{Y}\PY{l+s+s1}{\PYZsq{}}\PY{p}{)}
\end{Verbatim}

 \begin{Verbatim}[commandchars=\\\{\}]
{\color{incolor}In [{\color{incolor}27}]:} \PY{p}{(}\PY{n}{pure\PYZus{}premium} \PY{o}{+} \PY{l+m+mi}{134}\PY{p}{)} \PY{o}{*} \PY{l+m+mf}{1.15}
\end{Verbatim}

 \begin{Verbatim}[commandchars=\\\{\}]
{\color{outcolor}Out[{\color{outcolor}27}]:} 448.822958583028
\end{Verbatim}

        \fullline


    \textbf{Question 10:} Do you think an insurance company should use car
brand, gender, and age to determine a policyholder's premium? What other
variables can you think of that might have an impact on expected loss,
and why?

\emph{Yes, insurance companies should use car brand, gender, and age to
determine a policyholder's premium. Seeing as car insurance is mandatory
for car ownership, lower risk drivers who have less of a chance of
needing to make a claim (and therefore need less in claims) should not
have to help cover for more accident-prone drivers covered by the same
company.}





    \end{document}
